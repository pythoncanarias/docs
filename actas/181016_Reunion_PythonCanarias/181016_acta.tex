\documentclass[a4paper, 12pt]{article}

\usepackage[utf8]{inputenc}
\usepackage[spanish]{babel}
\usepackage{float}
\usepackage{hyperref}
\usepackage{tocloft}
\usepackage{titling}
\usepackage{eurosym}
\usepackage{bookmark}

\renewcommand{\labelitemi}{$\bullet$}
\renewcommand{\labelitemii}{$\circ$}
\renewcommand{\labelitemiii}{--}
\renewcommand{\cftsecleader}{\cftdotfill{\cftdotsep}}
\newcommand{\specialcell}[2][c]{%
    \begin{tabular}[#1]{@{}c@{}}#2\end{tabular}
}

\setlength{\droptitle}{-10em}

\hypersetup{
    colorlinks=true,
    linkcolor=black,
    filecolor=magenta,      
    urlcolor=cyan,
}

\hyphenation{Python-Canarias}
\hyphenation{Py-Day}
\hyphenation{Py-thon}

\title{\huge \textbf{Acta de reunión} \\ \textit{Python Canarias}}
\date{\textbf{16 de octubre de 2018}}
\author{
    Alberto Morales Díaz \and
    Héctor Figueras Hernández \and
    Jesús Miguel Torres Jorge \and
    Juan Ignacio Rodríguez de León\ \and 
    Lucas Grillo Lorenzo \and
    Raúl Marrero Rodríguez \and
    Sara Báez García \and
    Sergio Delgado Quintero
}

\begin{document}

\renewcommand{\contentsname}{Orden del día}

\maketitle

En \textit{San Cristóbal de La Laguna}, siendo las \textit{17:00h} de la fecha arriba indicada se reúnen en la \textit{Escuela Técnica Superior de Ingeniería Informática} los miembros de Python Canarias arriba indicados, a fin de tratar el siguiente orden del día:

\tableofcontents

\section{Lectura y aprobación, si procede, del acta anterior}

Se aprueba el acta anterior.

\section{Cuestiones organizativas sobre PyDay 2018}

\begin{itemize}
    \item Se ha solicitado presupuesto a varias empresas de todo el \textit{merchandising}/\textit{fungible} que se necesita para el evento. En función de lo que nos contesten se tomará una decisión al respecto. En este apartado se incluyen:
    \begin{itemize}
        \item Camisetas.
        \item Bolígrafos.
        \item Flyers.
        \item Roll-up.
        \item Pegatinas.
        \item Bolsas para \textit{welcome-pack}.
    \end{itemize}
    \item Se acuerda solicitar a Python España el \textit{flyer} de la última \texttt{\#PyConES18} con el objeto de tomar ideas para nuestro flyer.
    \item El dato que manejamos en relación al \textit{número de bolígrafos} que aporta Mutua Tinerfeña es de 100. Si no diera para repartir 1 bolígrafo por asistente en los welcome-pack entonces los dejaríamos en una mesa para que cada persona coja si le interesa.
    \item En relación a la \textit{megafonía} para los ponentes, se ve que no es necesario preocuparnos de esto. El Aula Magna disponía de sistema de megafonía y las aulas 11 y 12 no son tan grandes para necesitar megafonía.
    \item Se acuerda averiguar las dimensiones de las mesas que ofreceremos a los patrocinadores.
    \item Cabe la posibilidad de incorporar al fotógrafo del evento en el grupo de voluntariado. Se preguntará por ello.
    \item Finalmente se decide que el alumnado del \textit{CIFP César Manrique} realice la grabación y edición de las charlas.
    \item Se plantea la posibilidad de que \textit{Kreitek} ofrezca algún bono para el sorteo del final del evento.
    \item En relación a la grabación de las charlas se acuerda preguntar a Demetrio (profesor del \textit{CIFP César Manrique}) sobre la grabación de las pantallas de los ponentes.
    \item También se hace necesario preguntar a los ponentes si van a hacer \textit{live-coding}, cuestión a tener en cuenta a la hora de la grabación de las charlas.
    \item El resto de acuerdos organizativos se trasladan al \href{https://trello.com/b/HrpdbSYG/pyday-2018}{tablero correspondiente en el Trello de la asociación}.
\end{itemize}


\section{Informe sobre la asociación}

\begin{itemize}
    \item El pasado 10 de octubre la asesoría (\textit{Taoro Consultores}) presentó la subsanación en los estatutos que había sido requerida por el registro de asociaciones canarias.
    \item Se confirma que \textit{Héctor Álvarez} ha accedido a ser el nuevo secretario de la asociación en sustitución de Víctor Suárez. Se dejará constancia de este cambio en la primera junta de socios.
    \item Héctor Álvarez renovó el dominio \texttt{pythoncanarias.es} el pasado 15 de octubre.
    \item Se ha creado un documento de \textit{deudas} en el Drive de la asociación, con lo que va quedando pendiente de pagar a las personas del CORE.
\end{itemize}

\section{Ruegos y preguntas}

No hay ruegos ni preguntas.

% ================================================================================================

\vspace{1cm}
\hrule
\vspace{3mm}

Una vez tratados todos los puntos, se levanta la sesión cuando son las \textit{18:30h} en lugar y fecha arriba indicados.

\begin{flushright}
El secretario

Sergio Delgado Quintero
\end{flushright}

\end{document}
