\documentclass[a4paper, 12pt]{article}

\usepackage[utf8]{inputenc}
\usepackage[spanish]{babel}
\usepackage{float}
\usepackage{hyperref}
\usepackage{tocloft}
\usepackage{titling}

\renewcommand{\labelitemi}{$\bullet$}
\renewcommand{\labelitemii}{$\circ$}
\renewcommand{\labelitemiii}{--}
\renewcommand{\cftsecleader}{\cftdotfill{\cftdotsep}}

\setlength{\droptitle}{-10em}

\hypersetup{
    colorlinks=true,
    linkcolor=black,
    filecolor=magenta,      
    urlcolor=cyan,
}

\title{\huge \textbf{Acta de reunión} \\ \textit{Python Canarias}}
\date{\textbf{21 de marzo de 2018}}
\author{Juan Ignacio Rodríguez de León\ \and Sergio Delgado Quintero \and Víctor Suárez García \and José Lucas Grillo Lorenzo \and Alberto Morales Díaz}

\begin{document}

\renewcommand{\contentsname}{Orden del día}

\maketitle

En \textit{San Cristóbal de La Laguna}, siendo las \textit{18:00h} de la fecha arriba indicada se reúnen en la \textit{cafetería Nivaria} los miembros de Python Canarias arriba indicados, a fin de tratar el siguiente orden del día:

\tableofcontents

\section{Certificados del PyDay 2017}

\begin{itemize}
    \item Los certificados del PyDay 2017 han sido diseñados y maquetados por Juan Ignacio. Se decide una de las dos texturas de papel que trae Juan Ignacio para el acabado final del documento.
    \item Los certificados se crearán en formato PDF y papel. Se enviarán en PDF a los ponentes del PyDay 2017 y se les avisará de que pueden recogerlo en papel, posiblemente en la sede de Kreitek.
    \item De cara a futuras ediciones del PyDay, se acuerda que los certificados se harán digitales con una URL en la que se puedan validar, desarrollando algún módulo propio en la web de Python Canarias.
\end{itemize}

\section{PyBirras 2018}

\begin{itemize}
    \item Lucas comenta que ha hablado con el \href{http://www.equipopara.org/}{Equipo PARA} para celebrar la próxima edición de PyBirras y le han propuesto el 6 de abril como fecha disponible.
    \item De cara a cubrir los gastos del camarero del Equipo PARA, se acuerda que la entrada sea gratuita, pero incentivar a los asistentes a tomar alguna consumición, y en caso de que faltara dinero para cubrir gastos, lo pondría Python Canarias.
    \item Se acuerda que la hora de comienzo sean las 19:00h.
    \item Las charlas con las que se puede contar actualmente -- que tendrán una duración de 15-20 minutos -- son las siguientes:
    \begin{itemize}
        \item Ray:``Seguridad/Vulnerabilidades con Python``
        \item Juan Ignacio: ``Generación de escenarios en Blender con Python``
        \item Víctor: ``Cómo hacer un proyecto opensource con Python y no morir en el intento``
    \end{itemize}
    \item Una vez que se tengan cerradas las charlas, lugar, hora y otros detalles, se publicará una noticia en la web. Además Juan Ignacio diseñará un cartel para el evento y Sergio tuiteará sobre ello.
\end{itemize}

\section{Micropython 2018}

\begin{itemize}
    \item Se propone el 2 de junio como fecha de realización de Micropython 2018.
    \item Queda pendiente hablar con Javier (Brok-Air) para ver si podemos volver a usar sus instalaciones, y también queda pendiente hablar con Ruymán (K-Electrónica) para ver si colabora con nosotros con respecto a los kits.
\end{itemize}

\section{PyDay 2018}

\begin{itemize}
    \item Se propone el 10 de noviembre como fecha de realización del PyDay 2018.
    \item Se plantean distintos lugares para la celebración del PyDay 2018:
    \begin{itemize}
        \item Museo de la Ciencia y el Cosmos.
        \item Universidad de La Laguna.
            \begin{itemize}
                \item Facultad de periodismo.
                \item Facultad de bellas artes.
                \item Facultad de física.
            \end{itemize}
        \item CIFP César Manrique.
        \item Instituto Astrofísico de Canarias.
        \item Cabildo de Tenerife.
            \begin{itemize}
                \item Auditorio de Tenerife.
                \item Aceleradora en dársena del puerto.
            \end{itemize}
        \item Teatro Aguere.
    \end{itemize}
    \item Se acuerda hacer talleres el viernes por la tarde y un único track de charlas durante todo el sábado (mañana y tarde).
    \item Se plantea la posibilidad de ``alquilar`` un servicio de conexión a internet, de tal forma que no tengamos que depender de terceros para navegar. Una opción puede ser la empresa que se contrata para el evento Tecnológica Santa Cruz. Queda pendiente hacer la consulta de prestaciones y precios.
    \item Sergio aclara que es fundamental tener una sala con capacidad para más de 100 personas, en disposición tipo grada y con pantalla de proyección de gran formato, debido a que la mayoría de presentaciones enseñan código, y en otro caso, sería muy difícil de poder visualizar para el público sentando a una cierta distancia.
\end{itemize}

\section{Ruegos y preguntas}

\begin{itemize}
    \item Alberto aclara que el dominio \url{pythoncanarias.es} fue registrado por Héctor Álvarez en OVH Hispano. Debido a ello se debe de contar con un hosting mínimo de 10MB.
\end{itemize}

\vspace{1cm}
\hrule
\vspace{3mm}

Una vez tratados todos los puntos, se levanta la sesión cuando son las \textit{19:30h} en lugar y fecha arriba indicados.

\begin{flushright}
El secretario

Sergio Delgado Quintero
\end{flushright}


\end{document}
