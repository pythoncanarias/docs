\documentclass[a4paper, 12pt]{article}

\usepackage[T1]{fontenc}
\usepackage[utf8]{inputenc}
\usepackage[spanish]{babel}
\usepackage{float}
\usepackage{hyperref}
\usepackage{tocloft}
\usepackage{titling}
\usepackage{eurosym}
\usepackage{bookmark}

\renewcommand{\labelitemi}{$\bullet$}
\renewcommand{\labelitemii}{$\circ$}
\renewcommand{\labelitemiii}{--}
\renewcommand{\cftsecleader}{\cftdotfill{\cftdotsep}}
\newcommand{\specialcell}[2][c]{%
    \begin{tabular}[#1]{@{}c@{}}#2\end{tabular}
}

\setlength{\droptitle}{-10em}

\hypersetup{
    colorlinks=true,
    linkcolor=black,
    filecolor=magenta,      
    urlcolor=cyan,
}

\hyphenation{Python-Canarias}
\hyphenation{Py-Day}
\hyphenation{Py-thon}

\title{\huge \textbf{Acta de reunión} \\ Junta Directiva \\ \textit{Python Canarias}}
\date{\textbf{25 de noviembre de 2019}}
\author{
    Héctor Álvarez Alonso \and
    Israel Santana Alemán \and
    Juan Ignacio Rodríguez de León \and 
    Sergio Delgado Quintero
}

\begin{document}

\renewcommand{\contentsname}{Orden del día}

\maketitle

Vía \textit{Google Hangouts}, siendo las \textit{20:00h (Atlantic/Canary)} de la fecha arriba indicada se reúnen los miembros de la junta directiva de Python Canarias arriba indicados, a fin de tratar el siguiente orden del día:

\tableofcontents

\section{Lectura y aprobación, si procede, del acta anterior}

Se aprueba el acta anterior.

\section{Renovación de miembros de la junta directiva}

Dada la inactividad de ciertos miembros de la junta directiva, se decide notificar a estas personas que ratifiquen su compromiso con la asociación, o en caso contrario, se den de baja como miembros del órgano de representación. Las personas afectadas por esta decisión son los cuatro vocales, a saber:

\begin{itemize}
    \item Jaime Iván Juanes Prieto.
    \item José Lucas Grillo Lorenzo.
    \item Víctor Raúl Ruiz Ruiz.
    \item Víctor Suárez García.
\end{itemize}

De cara a la futura renovación de la junta directiva, se cree necesario la inminente puesta en marcha de un procedimiento para registrar socios/as. Hace tiempo que se viene desarrollando un procedimiento automático de alta de socios/as, pero aún no ha finalizado su implementación. Es por ello que se decide establecer un procedimiento manual mediante transferencia bancaria. Se habilitará una página en \textit{pythoncanarias.es} para la gestión de nuevos socios/as.

\section{Análisis del CORE y grupos para gestión de eventos}

En su día, cuando se montó el CORE, tenía el objetivo de servir como apoyo al desarrollo de eventos. Por ello se decidió crear un grupo de \textit{Telegram} como punto de encuentro y plataforma de comunicación. A día de hoy, después del crecimiento de la asociación y el desarrollo de eventos, la comunidad ha crecido y, con ella, las personas con las que podemos contar para la gestión de eventos.

En estos últimos tiempos el CORE ha tenido una gran inactividad y consideramos conveniente reestructurarlo y darle otra finalidad. Así, este grupo pasará a ser el grupo de los socios y socias de Python Canarias. Se dará un tiempo prudencial a los miembros para que se den de alta como socios y socias.

\section{Futuras acciones}

En este punto se toman los siguientes acuerdos:
\begin{itemize}
    \item Formar grupos \textit{ad hoc} para cada evento en concreto, contando con personas de la comunidad que quieran colaborar e implicarse en el desarrollo del mismo.
    \item Alternar la realización cada año del PyDay entre Tenerife y Gran Canaria, como evento insignia de la asociación.
    \item Intentar montar eventos más modestos en islas menores.
    \item Aunque los eventos sean pequeños o minoritarios, cobrar siempre algo en la entrada, para evitar las inscripciones ficticias.
    \item Estudiar la viabilidad de eventos tipo \textit{OpenSpace}.
\end{itemize}

\section{Ruegos y preguntas}

No hay ruegos ni preguntas.

% ================================================================================================

\vspace{1cm}
\hrule
\vspace{3mm}

Una vez tratados todos los puntos, se levanta la sesión cuando son las \textit{21:30h} en lugar y fecha arriba indicados.

\begin{flushright}
El secretario

Sergio Delgado Quintero
\end{flushright}

\end{document}
