\documentclass[a4paper, 12pt]{article}

\usepackage[utf8]{inputenc}
\usepackage[spanish]{babel}
\usepackage{float}
\usepackage{hyperref}
\usepackage{tocloft}
\usepackage{titling}
\usepackage{eurosym}
\usepackage{bookmark}

\renewcommand{\labelitemi}{$\bullet$}
\renewcommand{\labelitemii}{$\circ$}
\renewcommand{\labelitemiii}{--}
\renewcommand{\cftsecleader}{\cftdotfill{\cftdotsep}}
\newcommand{\specialcell}[2][c]{%
    \begin{tabular}[#1]{@{}c@{}}#2\end{tabular}
}

\setlength{\droptitle}{-10em}

\hypersetup{
    colorlinks=true,
    linkcolor=black,
    filecolor=magenta,      
    urlcolor=cyan,
}

\hyphenation{Python-Canarias}
\hyphenation{Py-Day}
\hyphenation{Py-thon}

\title{\huge \textbf{Acta de reunión} \\ \textit{Python Canarias}}
\date{\textbf{27 de diciembre de 2018}}
\author{
    Alejandro Rodríguez Casañas \and
    Alejandro Samarín Pérez \and
    Héctor Figueras Hernández \and
    Iván Rodríguez Méndez \and
    Juan Ignacio Rodríguez de León\ \and 
    Lucas Grillo Lorenzo \and
    Raúl Marrero Rodríguez \and
    Sara Báez García \and
    Sergio Delgado Quintero
}

\begin{document}

\renewcommand{\contentsname}{Orden del día}

\maketitle

En \textit{La Orotava}, siendo las \textit{17:00h} de la fecha arriba indicada se reúnen en la \textit{MyPlace Coworking Tenerife} los miembros de Python Canarias arriba indicados, a fin de tratar el siguiente orden del día:

\tableofcontents

\section{Lectura y aprobación, si procede, del acta anterior}

Se aprueba el acta anterior.

\section{Evaluación del PyDay 2018}

Analizando las encuestas de satisfacción y la propia experiencia organizativa, se detallan a continuación las \textit{propuestas de mejora} de cara a la próxima edición:

\subsection*{Vídeos de las charlas}

\begin{itemize}
    \item Fijar una fecha en la que los vídeos deberían estar montados.
    \item No esperar al finalizar la edición de todos los vídeos para entregarlos. Al contrario, interesa hacer entregas parciales para ir detectando posibles errores y además publicarlos a la mayor brevedad posible.
    \item Avisar, a priori, de los detalles de maquetación importantes: ubicación, nombres de ponentes, nombres de charlas, estructura, \ldots
\end{itemize}

\subsection*{Espacios de las charlas}

\begin{itemize}
    \item Estudiar la posibilidad de hacerlo en la \textit{ESIT} de la \textit{ULL}. A nivel de proyección ofrece ventajas con respecto al Edificio de Física y Matemáticas, ya que hay mayor oscuridad en las aulas y se ve mucho mejor.
    \item Buscar la forma de oscurecer las aulas y mejorar los proyectores.
    \item Incorporar sistemas de megafonía incluso en aulas secundarias.
\end{itemize}

\subsection*{Selección de charlas}

\begin{itemize}
    \item Aceptar menos charlas y buscar mayor calidad en las mismas.
    \item Añadir un track de talleres.
    \item Sería interesante ofertar un \textit{Taller de introducción a la Programación con Python}.
    \item Para los talleres es necesario que cada asistente traiga su portátil.
    \item Dar la posibilidad de \textit{charlas lightning} de corta duración para aquellos ponentes que no han sido seleccionados.
    \item En una primera fase de \textit{Call for Papers} vale con pedir una reseña de la charla, pero una vez seleccionada, si procedería preguntar cuestiones más concretas, con el objetivo de dar una información más completa al asistente.
    \item Requerir un índice para aquellas charlas finalmente seleccionadas.
\end{itemize}

\subsection*{Otras cuestiones}

\begin{itemize}
    \item Dar pautas a ponentes sobre la presentación:
    \begin{itemize}
        \item Para código: letra blanca sobre fondo oscuro.
        \item Tamaño de letra grande.
        \item Plantilla de presentación (como sugerencia).
    \end{itemize}
    \item Separar más la comida de celiacos de la comida del resto de asistentes, por aquello de la posible contaminación.
    \item Ser más previsores en la adquisición del \textit{merchandising}.
    \item Cada año se puede compatibilizar \textit{PyDay}s en Tenerife y Gran Canaria. Lo único es separarlos suficientemente en el tiempo.
    \item Tras cuadrar las cuentas con Python España, el evento quedó con un superhábit de 59,38\euro que Python España transferirá a Python Canarias.
\end{itemize}

\section{Próximo evento}

En \textbf{Tenerife} se acuerda realizar un \textit{evento tecnológico para mujeres}.

\begin{itemize}
    \item La fecha acordada es el \textit{23 de marzo de 2019}.
    \item Se decide hacer un sondeo previo entre los grupos tecnológicos de mujeres para ver el interés que despierta. Las tres opciones que proponemos son:
    \begin{itemize}
        \item Taller de \textit{Django Girls}.
        \item Taller de \textit{introducción a la programación con Python}.
        \item Taller de \textit{Flask} + (tu \textit{front-end}).
    \end{itemize}
    \item Se barajan distintos lugares para su celebración:
    \begin{itemize}
        \item Coworking de INTECH en Dársena.
        \item Alguna biblioteca municipal.
        \item IAC (esta opción gusta porque tendría un espacio para dejar a niños/niñas).
    \end{itemize}
    \item Se propone cobrar \textit{5\euro} como precio de entrada.
\end{itemize}

En \textbf{Gran Canaria} hay intención de realizar\footnote{No necesariamente en este orden}:

\begin{itemize}
    \item Taller de micropython.
    \item PyDay.
    \item Jornada informal de Python.
\end{itemize}

\section{Información sobre la asociación}

\begin{itemize}
    \item Ya se ha dado de alta la asociación tanto en el registro de asociaciones canarias como en la \textit{AEAT}.
    \item Ya se ha conseguido la creación de la cuenta bancaria de la asociación en \textit{OpenBank}.
    \item Juan Ignacio asistirá el día \textit{28 de diciembre de 2018} a una reunión de la \textit{Tenerife Tech Alliance}.
    \item Se ha creado un grupo de Telegram para los miembros de la junta directiva.
    \item Próximamente se hará una asamblea general para aprobar el cambio de tesorero:
    \begin{itemize}
        \item Nuevo tesorero: \textit{Héctor Álvarez Alonso}.
        \item Nuevo vocal: \textit{Víctor Suárez García}.
    \end{itemize}
    \item Se está en conversaciones con \textit{Leopoldo Acosta Sánchez} -- director de la ESIT -- y con \textit{Francisco de Sande} -- vicerrector de nuevas tecnologías de la ULL --, con el objetivo de intentar conseguir un espacio en la ESIT para Python Canarias.
\end{itemize}

\section{Ruegos y preguntas}

No hay ruegos ni preguntas.

% ================================================================================================

\vspace{1cm}
\hrule
\vspace{3mm}

Una vez tratados todos los puntos, se levanta la sesión cuando son las \textit{18:50h} en lugar y fecha arriba indicados.

\begin{flushright}
El secretario

Sergio Delgado Quintero
\end{flushright}

\end{document}
